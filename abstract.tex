\documentclass[12pt]{eccm-ecfd_abstract}

\title{FAST DIVERGENCE-CONFORMING REDUCED ORDER MODELS FOR FLOW PROBLEMS}

\author{Eivind Fonn$^{1}$, Harald van Brummelen$^{2}$, Trond Kvamsdal$^{3}$, Adil
  Rasheed$^{1}$ and Muhammad S. Siddiqui$^{3}$}

\address{$^{1}$ SINTEF Digital, Kl{\ae}buveien 153 7037 Trondheim NORWAY, eivind.fonn@sintef.no
\and
$^{2}$ Eindhoven University of Technology, Department of Mechanical Engineering
\and
$^{3}$ NTNU, Department of Mathematical Sciences}

\begin{document}

\noindent {\bf Keywords}: {\it Reduced Basis Methods, Reduced Order Models,
Proper Orthogonal Decomposition, Navier-Stokes, Divergence-Conforming,
Isogeometry}
\vskip0.5cm

Numerical methods and tools for simulating flow around complex geometries have
evolved rapidly in recent years. However, their usage usually requires access to
high-performance computating facilities, which is not always feasible. New
methods are required for an ever-increasing demand for computationally efficient
desktop tools usable in real-time control, optimization and management. One such
solution is dimensionality reduction through Reduced Order Modelling (ROM) based
on Proper Orthogonal Decomposition (POD).\cite{Quarteroni2016rbm}

We detail in this work an experiment involving two-dimensional stationary flow
around a NACA0015 airfoil at various velocities and angles of attack. Two
different methods are proposed: one run-of-the-mill reduced basis method based
on a conventional Taylor-Hood high-fidelity solver, and one based on a
divergence-conforming isogeometric flow solver. The reduced bases were
constructed with Proper Orthogonal Decomposition (POD) and enhanced with
supremizers\cite{Ballarin2015ssp} to stabilize the pressure field. We show how
the latter method produces a fully divergence-free reduced basis, and how this
property influences the linear systems by creating three-by-three block
triangular matrices, a structure that can be exploited for significant speedups
over the traditional matrix structure arising from the regular method.

\bibliography{common/references}
\bibliographystyle{plain}

\end{document}


